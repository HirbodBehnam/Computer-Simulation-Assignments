% !TEX program = xelatex
\documentclass[]{article}
\usepackage{commons/course}

\begin{document}
\printheader

\section{}
به صورت خلاصه می‌توان گفت که اعداد تصادفی
\lr{Pseudo Randoms}
واقعا بر حسب شانس بدست نمی‌آیند و واقعی نیستند ولی اعدادی هستند که می‌توان آنها را به عنوان اعداد
تصادفی در نظر گرفت ولی در عمل صرفا یک دنباله از اعداد هستند که در صورتی که چندین‌تای آنها را داشت
می‌توان کل دنباله را بدست آورد. از طرفی دیگر
\lr{random generator}ها
بر اساس اتفاقات رندوم در طبیعت مثل
\lr{atmospheric noise}
کار می‌کنند. همچنین جا دارد که به
\lr{CRNG}ها
نیز اشاره بکنم که برای ساخت اعداد تصادفی مناسب رمزنگاری استفاده می‌شوند.

\section{}
\lr{Bathtub curve}
یک منحنی است که احتمال خراب شدن قطعه‌ای را بر حسب زمان نشان می‌دهد. در ابتدای زمان ساخت برخی از
قطعات به صورت ذاتی خراب هستند و کمی بعد از کار کردن خراب می‌شوند. سپس قطعات سالم می‌مانند تا زمان زیادی و بعد از کارکرد
بسیار زیاد باز احتمال خراب شدن بیشتر می‌شود.
به همین منظور کارخانه‌ها بعد از تولید یک قطعه معمولا کمی آنرا مورد استفاده قراره می‌دهند که از این خرابی زودرس جلوگیری کنند.

\section{}
\begin{align*}
    E[X] &= \sum x P(X = x)\\
    &= \sum_{k \mathop = 0}^n k {n \choose k} p^k q^{n - k}\\
    &= \sum_{k \mathop = 1}^n k {n \choose k} p^k q^{n - k}\\
    &= n p \sum_{k \mathop = 1}^n {{n - 1} \choose {k - 1}} p^{k - 1} q^{(n - 1) - (k - 1)}\\
    &\stackrel{m=n-1, j=k-1}{=} = n p \sum_{j \mathop = 0}^m {m \choose j} p^j q^{m - j}\\
    &= np
\end{align*}

\section{}
\begin{gather*}
    E[X] = a + (b - a) \frac{\beta_1}{\beta_1 + \beta_2}\\
    V[X] = (b - a)^2 \frac{\beta_1 \beta_2}{(\beta_1+\beta_2)^2 (\beta_1 + \beta_2 + 1)}
\end{gather*}

\section{}
نکته‌ای که وجود دارد این است که زمان یک متغیر پیوسته است و برای همین به انتگرال نیاز داریم. همچنین لزومی
ندارد که متغیر ما به صورت ثابت باشد و می‌تواند حتی نسبت به زمان تغییر نیز بکند برای همین ما نیاز
به همچین عبارتی داریم.

\section{}
% a = np.array([[0.3, 0.2, 0.5],[0.4,0.3,0.3],[0.3,0.4,0.3]])
% np.array([1, 0, 0]) @ a @ a @ a @ a @ a @ a @ a @ a @ a @ a
\begin{gather*}
    \begin{bmatrix}
        1 & 0 & 0
    \end{bmatrix}
    \times
    \left(
        \begin{bmatrix}
            0.3 & 0.2 & 0.5\\
            0.4 & 0.3 & 0.3\\
            0.3 & 0.4 & 0.3
        \end{bmatrix}
    \right)^\infty
    =
    \begin{bmatrix}
        0.3303 & 0.3035 & 0.366
    \end{bmatrix}\\
    \begin{bmatrix}
        0 & 1 & 0
    \end{bmatrix}
    \times
    \left(
        \begin{bmatrix}
            0.3 & 0.2 & 0.5\\
            0.4 & 0.3 & 0.3\\
            0.3 & 0.4 & 0.3
        \end{bmatrix}
    \right)^\infty
    =
    \begin{bmatrix}
        0.3303 & 0.3035 & 0.366
    \end{bmatrix}\\
    \begin{bmatrix}
        0 & 0 & 1
    \end{bmatrix}
    \times
    \left(
        \begin{bmatrix}
            0.3 & 0.2 & 0.5\\
            0.4 & 0.3 & 0.3\\
            0.3 & 0.4 & 0.3
        \end{bmatrix}
    \right)^\infty
    =
    \begin{bmatrix}
        0.3303 & 0.3035 & 0.366
    \end{bmatrix}
\end{gather*}
مشاهده می‌شود که تمامی جواب‌ها یکسان شد!

\section{}
دلیل این موضوع این است که جمع احتمالاتی که از یک یال راس خارج می‌شوند عملا باید برابر با یک باشد
که بالاخره بعد از آن
\lr{tansit}
جایی برویم.

\section{}

\section{}
از آنجایی که
$\rho$ همان \lr{utilization}
است، پس
$1 - \rho$
می‌شود احتمال خالی یا بیکار بودن سیستم. احتمال اضافه شدن فرد دیگری به سیستم بدون رفتن بیرون کسی دیگری
نیز برابر
$rho$
است. پس در نهایت داریم:
$P_n = (1 - \rho) \rho^n$

\end{document}
