\section*{سوال 3}
در ابتدا بررسی می‌کنیم که این توزیع دقیقا چیست. این توزیع به ما می‌گوید که اگر یک رویداد داشته باشیم که
احتمال موفقیت در آن
$p$
باشد چند بار آزمایش شکست می‌خورد تا اینکه
$k$
بار موفقیت آمیز باشد. به همین منظور آزمایش را بدین صورت بیان می‌کنیم:
اعداد تصادفی بین ۰ تا ۱ تولید می‌کنیم و می‌شماریم که چند بار شسکت خورده‌ایم. زمانی که
$k$
تا موفقیت دیدیم،‌ تعداد شکست‌ها عدد مورد نظر ما خواهد بود. زمانی که به تعداد
$k$
موفقیت هم می‌رسیم شمارنده را ریست می‌کنیم.

پس به عنوان مثال در اعداد داده شده شروع می‌کنیم و جدول زیر را تشکیل می‌دهیم:
\begin{latin}
    \centering
    \begin{tabular}{|c|c|c|}
        \hline
        RNG & Fail & Win\\
        \hline
        0.81 & 1 & 0\\
        0.65 & 2 & 0\\
        0.72 & 3 & 0\\
        0.95 & 4 & 0\\
        0.2 & 4 & 1\\
        0.86 & 5 & 1\\
        0.4 & 5 & 2\\
        0.75 & 6 & 2\\
        0.35 & 6 & 3\\
        \hline
    \end{tabular}
\end{latin}
عدد تولید شده ۶ است.