\section*{سوال 4}
اعداد را مرتب می‌کنیم و جدول زیر را تشکیل می‌دهیم:
\begin{latin}
\centering
\begin{tabular}{|c|c|c|c|c|c|c|c|c|c|c|}
    \hline
    $R_i$ & 0.02 & 0.09 & 0.15 & 0.25 & 0.31 & 0.43 & 0.6 & 0.8 & 0.85 & 0.95\\
    \hline
    $\frac{i}{N}$ & 0.1 & 0.2 & 0.3 & 0.4 & 0.5 & 0.6 & 0.7 & 0.8 & 0.9 & 1\\
    \hline
    $\frac{i}{N} - R_i$ & 0.08 & 0.11 & 0.15 & 0.15 & 0.19 & 0.17 & 0.1 & 0 & 0.05 & 0.05\\
    \hline
    $\frac{i-1}{N}$ & 0 & 0.1 & 0.2 & 0.3 & 0.4 & 0.5 & 0.6 & 0.7 & 0.8 & 0.9\\
    \hline
    $R_i - \frac{i - 1}{N}$ & 0.02 & - & - & - & - & - & 0 & 0.1 & 0.05 & 0.05\\
    \hline
\end{tabular}
\end{latin}
حال داریم:
\begin{gather*}
    D^+ = \max\left(\frac{i}{N}-R_i\right) = 0.19\\
    D^- = \max\left(R_i - \frac{i}{N}\right) = 0.1\\
    D = \max\left(0.19, 0.1\right) = 0.19
\end{gather*}
حال با توجه به جدول داخل اسلاید 6، ستون
$D_{0.05}$
و ردیف
\lr{degree of freedom = 10}
عدد داخل ان نوشته شده برابر
$0.41$
است. از آنجا که
$D < 0.41$
است نمی‌توان
\lr{null hypothesis}
را رد کرد و می‌گوییم که اعداد یونیفرم هستند.