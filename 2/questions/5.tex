\smalltitle{سوال 5}
در ابتدا
\lr{markov chain}
اینکه چگونه می‌توان به رشته‌ی مورد نظر سوال رسید را رسم می‌کنیم.
\begin{latin}
    \begin{center}
        \begin{tikzpicture}[->, >=stealth', auto, semithick, node distance=3cm]
        \tikzstyle{every state}=[fill=white,draw=black,thick,text=black,scale=1]
        \node[state,initial]    (epsilon) {$q_\epsilon$};
        \node[state]    (T)[right of=epsilon]   {$q_{T}$};
        \node[state]    (TH)[right of=T]  {$q_{TH}$};
        \node[state]    (THT)[right of=TH] {$q_{THT}$};
        \path
        (epsilon) edge[loop above]        node{$1/3$}	(epsilon)
        (epsilon) edge[bend left,above]	node{$2/3$}	(T)
        (T) edge[bend left,above]	node{$1/3$}	(TH)
        (T) edge[loop above]	node{$2/3$}	(T)
        (TH) edge[bend left,below]	node{$2/3$}	(THT)
        (TH) edge[bend left,below]	node{$1/3$}	(epsilon)
        (THT) edge[loop above]	node{$1$}	(THT);
        \end{tikzpicture}
    \end{center}
\end{latin}
معادلات زمان رسیدن به هر
\lr{node}
را می‌نویسیم:
\begin{align*}
    h(q_\epsilon) &= 1 + \frac{1}{3} h(q_\epsilon) + \frac{2}{3} h(q_{T})\\
    h(q_{T}) &= 1 + \frac{1}{3} h(q_{TH}) + \frac{2}{3} h(q_{T})\\
    h(q_{TH}) &= 1 + \frac{2}{3} h(q_{THT}) + \frac{1}{3} h(q_{\epsilon})\\
    h(q_{THT}) &= 0
\end{align*}
پس داریم:
\begin{align*}
    h(q_\epsilon) &= 1 + \frac{1}{3} h(q_\epsilon) + \frac{2}{3} h(q_{T})\\
    h(q_{T}) &= 1 + \frac{1}{3} (1 + \frac{1}{3} h(q_{\epsilon})) + \frac{2}{3} h(q_{T})\\
    h(q_{TH}) &= 1 + \frac{1}{3} h(q_{\epsilon})\\
    h(q_{THT}) &= 0
\end{align*}
و در نتیجه:
\begin{align*}
    h(q_\epsilon) &= 1 + \frac{1}{3} h(q_\epsilon) + \frac{2}{3} (4 + \frac{1}{3} h(q_{\epsilon}))\\
    h(q_{T}) &= 3 + 1 + \frac{1}{3} h(q_{\epsilon})\\
    h(q_{TH}) &= 1 + \frac{1}{3} h(q_{\epsilon})\\
    h(q_{THT}) &= 0
\end{align*}
در نهایت داریم
$h(q_\epsilon) = 8.25$