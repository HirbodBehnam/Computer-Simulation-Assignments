\smalltitle{سوال 4}
\\\noindent
در ابتدا زنجیره را رسم می کنیم:
\begin{latin}
    \begin{center}
        \begin{tikzpicture}[->, >=stealth', auto, semithick, node distance=3cm]
        \tikzstyle{every state}=[fill=white,draw=black,thick,text=black,scale=1]
        \node[state,initial]    (A)       {No Symptom};
        \node[state]    (B)[right of=A]   {Symptom};
        \node[state]    (C)[right of=B]   {Dead};
        \path
        (A) edge[loop below]        node{$1/3$}	(A)
        (A) edge[bend left,below]	node{$1/3$}	(B)
        (A) edge[bend left,above]	node{$1/3$}	(C)
        (B) edge[loop below]    	node{$1/2$}	(B)
        (B) edge[bend left,below]	node{$1/2$}	(C)
        (C) edge[loop below]    	node{$1$}	(C);
        \end{tikzpicture}
    \end{center}
\end{latin}
حال ماتریس انتقال را بدست می‌آوریم.
\begin{gather*}
    P = \begin{bmatrix}
        \frac{1}{3} & \frac{1}{3} & \frac{1}{3}\\
        0 & \frac{1}{2} & \frac{1}{2} \\
        0 & 0 & 1
    \end{bmatrix}
    % P := <<1/3,0,0>|<1/3,1/2,0>|<1/3,1/2,1>>;
\end{gather*}
به کمک ماشین حساب می توان دید که
$P^n$
هر چه قدر که
$n$
را بزرگتر کنیم به
$\big[\begin{smallmatrix}
    0 & 0 & 1\\
    0 & 0 & 1 \\
    0 & 0 & 1
\end{smallmatrix}\big]$
نزدیک‌تر می‌شود. پس در صورتی که هر کدام از
$\big[\begin{smallmatrix}
    1 & 0 & 0
\end{smallmatrix}\big]$
یا
$\big[\begin{smallmatrix}
    0 & 1 & 0
\end{smallmatrix}\big]$
یا
$\big[\begin{smallmatrix}
    0 & 0 & 1
\end{smallmatrix}\big]$
را در ماتریس ضرب کنیم جواب آن برابر
$\big[\begin{smallmatrix}
    0 & 0 & 1
\end{smallmatrix}\big]$
می‌شود که نشان می‌دهد که بیمار در زمان بی‌نهایت می‌میرد.

برای زمان متوسط مرگ می‌توان از فرمول زیر استفاده کرد:
\begin{gather*}
    \sum_{i=1}^{\infty} i \times \big(\begin{bmatrix}
        1 & 0 & 0
    \end{bmatrix} P^i \begin{bmatrix}
        0 \\ 0 \\ 1
    \end{bmatrix}\big)
\end{gather*}
که به زبان ساده می‌شود احتمال مرگ بیمار بعد از
$i$
سیکل ضرب در
$i$
به ازای تمامی
$i$های
مختلف.


