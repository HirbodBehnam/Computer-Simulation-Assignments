\smalltitle{سوال 4}
در ابتدا زنجیره را رسم می کنیم:
\begin{latin}
    \begin{center}
        \begin{tikzpicture}[->, >=stealth', auto, semithick, node distance=3cm]
        \tikzstyle{every state}=[fill=white,draw=black,thick,text=black,scale=1]
        \node[state,initial]    (A)       {No Symptom};
        \node[state]    (B)[right of=A]   {Symptom};
        \node[state]    (C)[right of=B]   {Dead};
        \path
        (A) edge[loop below]        node{$1/3$}	(A)
        (A) edge[bend left,below]	node{$1/3$}	(B)
        (A) edge[bend left,above]	node{$1/3$}	(C)
        (B) edge[loop below]    	node{$1/2$}	(B)
        (B) edge[bend left,below]	node{$1/2$}	(C)
        (C) edge[loop below]    	node{$1$}	(C);
        \end{tikzpicture}
    \end{center}
\end{latin}
حال ماتریس انتقال را بدست می‌آوریم.
\begin{gather*}
    P = \begin{bmatrix}
        \frac{1}{3} & \frac{1}{3} & \frac{1}{3}\\
        0 & \frac{1}{2} & \frac{1}{2} \\
        0 & 0 & 1
    \end{bmatrix}
    % P := <<1/3,0,0>|<1/3,1/2,0>|<1/3,1/2,1>>;
\end{gather*}
به کمک ماشین حساب می توان دید که
$P^n$
هر چه قدر که
$n$
را بزرگتر کنیم به
$\big[\begin{smallmatrix}
    0 & 0 & 1\\
    0 & 0 & 1 \\
    0 & 0 & 1
\end{smallmatrix}\big]$
نزدیک‌تر می‌شود. پس در صورتی که هر کدام از
$\big[\begin{smallmatrix}
    1 & 0 & 0
\end{smallmatrix}\big]$
یا
$\big[\begin{smallmatrix}
    0 & 1 & 0
\end{smallmatrix}\big]$
یا
$\big[\begin{smallmatrix}
    0 & 0 & 1
\end{smallmatrix}\big]$
را در ماتریس ضرب کنیم جواب آن برابر
$\big[\begin{smallmatrix}
    0 & 0 & 1
\end{smallmatrix}\big]$
می‌شود که نشان می‌دهد که بیمار در زمان بی‌نهایت می‌میرد. همچنین می‌توان استدلال کرد که از آنجا که در صورتی که
در این حالت باشیم با احتمال یک به خودش بر می‌گردیم. پس در نهایت در زمانی که به این حالت برسیم
دیگر در خودش می‌مانیم و در نهایت با احتمال بزرگتر از یک از هر حالتی که شروع کنیم به این حالت می‌رسیم.

برای زمان متوسط مرگ می‌توان به صورت زیر عمل کرد:
\begin{align*}
    h(q_{NS}) &= 1 + \frac{1}{3} h(q_{NS}) + \frac{1}{3} h(q_{S}) + \frac{1}{3} h(q_{D})\\
    h(q_{S}) &= 1 + \frac{1}{2} h(q_{S}) + \frac{1}{2} h(q_{D})\\
    h(q_{D}) &= 0\\
    \implies\\
    h(q_{NS}) &= 1 + \frac{1}{3} h(q_{NS}) + \frac{1}{3} h(q_{S})\\
    h(q_{S}) &= 1 + \frac{1}{2} h(q_{S})\\
    h(q_{D}) &= 0\\
    \implies\\
    h(q_{NS}) &= 1 + \frac{1}{3} h(q_{NS}) + \frac{2}{3}\\
    h(q_{S}) &= 2\\
    h(q_{D}) &= 0\\
    \implies\\
    h(q_{NS}) &= \frac{5}{2}\\
    h(q_{S}) &= 2\\
    h(q_{D}) &= 0
\end{align*}


