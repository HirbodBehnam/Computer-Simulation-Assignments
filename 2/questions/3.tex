\smalltitle{سوال 3}
\begin{enumerate}
    \item \phantom{text} \begin{latin}
        \begin{center}
            \begin{tikzpicture}[->, >=stealth', auto, semithick, node distance=3cm]
            \tikzstyle{every state}=[fill=white,draw=black,thick,text=black,scale=1]
            \node[state,initial]    (A)       {$0$};
            \node[state]    (B)[right of=A]   {$1$};
            \node[state]    (C)[right of=B]   {$2$};
            \node[state]    (D)[right of=C]   {$\dots$};
            \path
            (A) edge[loop above]        node{$1-p$}	(A)
            (A) edge[bend left,above]	node{$p$}	(B)
            (B) edge[bend left,below]	node{$1-p$}	(A)
            (B) edge[bend left,above]	node{$p$}	(C)
            (C) edge[bend left,below]	node{$1-p$}	(B)
            (C) edge[bend left,above]	node{$p$}	(D)
            (D) edge[bend left,below]	node{$1-p$}	(C);
            \end{tikzpicture}
        \end{center}
    \end{latin}
    \item برای $P_0$ و $P_1$
    داریم:
    \begin{gather*}
        P_0 = P_0 (1 - p) + P_1 (1 - p)\\
        \implies P_0 - P_0 + P_0 p = P_1 (1 - p) \implies P_0 = \frac{1 - p}{p} P_1 
    \end{gather*}
    و برای بقیه‌ی حالات داریم:
    \begin{gather*}
        P_i = P_{i-1} p + P_{i+1} (1 - p)\\
    \end{gather*}
    به عنوان مثال برای حالت
    $P_1$ و $P_2$
    داریم:
    \begin{gather*}
        P_1 = P_{0} p + P_{2} (1 - p) = \frac{1 - p}{p} P_1 p + P_{2} (1 - p) = P_1 (1 - p) + P_{2} (1 - p)\\
        \implies P_1 p = P_{2} (1 - p) \implies P_1 = \frac{1 - p}{p} P_2
    \end{gather*}
    به صورت مشابه می‌توان نشان داد که
    \begin{gather*}
        P_i = \frac{1 - p}{p} P_{i+1} \implies P_{i+1} = \frac{p}{1 - p} P_i = (\frac{p}{1 - p})^{i} P_0
    \end{gather*}
    حال سعی می‌کنیم که
    $P_0$
    را پیدا کنیم. می‌دانیم که جمع تمامی احتمال‌ها برابر 1 است. پس داریم:
    \begin{gather*}
        \sum_{i = 0}^{\infty} (\frac{p}{1 - p})^{i} P_0 =  P_0 \sum_{i = 0}^{\infty} (\frac{p}{1 - p})^{i} = 1\\
        \implies P_0 = \frac{1}{\sum_{i = 0}^{\infty} (\frac{p}{1 - p})^{i}} \stackrel{\frac{p}{1 - p} < 1}{=} 1 - \frac{p}{1 - p} = \frac{1 - 2p}{1 - p}\\
        \implies P_i =  (\frac{p}{1 - p})^{i} \frac{1 - 2p}{1 - p} 
    \end{gather*}
    \item در قسمت قبل با توجه به اینکه
    $p < 0.5$ بود نتیجه گرفتیم که
    $\frac{p}{1 - p} < 1$
    است. اما در صورتی که
    $p > 0.5$
    باشد آنگاه داریم:
    $\frac{p}{1 - p} > 1$
    و می‌دانیم که جمع سری هندسی
    $\sum_{i=0}^{\infty} ar^i$
    برای
    $r > 1$
    تعریف نمی‌شود. پس باید
    $p < 0.5$ 
    باشد.
\end{enumerate}