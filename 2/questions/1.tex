\smalltitle{سوال 1}
از
\lr{NSPP}
استفاده می‌کنیم. فرض کنید به مدت
$T$
ثانیه حالت مذکور در سوال رخ می‌دهد که نرخ ما افزایش می‌یابد. بدون کاستی از مسئله فرض می‌کنیم که این اتفاق
در زمان 0 تا
$T$
می‌افتد و در زمان
$T$
تا 10 همان نرخ سابق وجود دارد.
در این صورت داریم:
\begin{gather*}
    \lambda(t) = \begin{cases}
        3 & 0 \le t < T \\
        2 & T \le t < 10
    \end{cases}
\end{gather*}
می‌دانیم که داریم
$\Lambda(t) = \int_{0}^{t} \lambda(s) ds$
پس نتیجه می‌شود:
\begin{gather*}
    \Lambda(t) = \begin{cases}
        \int_{0}^{t} 3 ds = 3t & 0 \le t < T \\
        \int_{0}^{T} 3 ds + \int_{T}^{t} 2 ds = T + 2 t & T \le t < 10
    \end{cases}\\
    \Lambda(10) = T + 20\\
    P(x = 5) = \frac{e^{-(T + 20)} (T + 20)^5}{5!}
\end{gather*}
حال به کمک یک انتگرال و احتمال شرطی به ازای مقدایر مختلف
$T$
از 0 تا 10 احتمال ذکر شده را حساب می‌کنیم. داریم:
\begin{gather*}
    \text{Answer} = \int_{0}^{10} P(x = 5 | T = t) P(T = t) dt
\end{gather*}
برای حساب کردن
$P(T = t)$
از
\lr{PDF}
توزیع گاما مطابق اسلاید‌ها استفاده می‌کنیم:
\begin{gather*}
    f_T(t) = \frac{2 \times 5}{(2 - 1)!} (2 \times 5 t)^{2 - 1} e^{-2 \times 5 \times t} = 100 t e^{-10t}
\end{gather*}
حال انتگرال مذکور به صورت زیر در می‌آید:
\begin{gather*}
    \text{Answer} = \int_{0}^{10} \frac{e^{-(t + 20)} (t + 20)^5}{5!} \times 100 t e^{-10t} dt \approx 0.0000475
\end{gather*}
\



